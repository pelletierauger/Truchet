%!TEX root = structure.tex

\section{Introduction}
Ceci est l'introduction à mon projet sur les pavages de Truchet.

\section{Réflexions diverses}

Je dois penser à mes systèmes de Truchet comme à des espaces mathématiques dans lesquels je peux me déplacer, un peu comme l'ensemble de Mandelbrot est un espace mathématique. Je peux créer des \textit{ensembles de Truchet} que mon système pourra visualiser à n'importe quelles valeurs.

Idéalement, il me faudrait pouvoir définir mes ensembles de Truchet ainsi :
Un groupe de 4 permutations : A, C, B, A, miroité horizontalement avec A, B, C, A, et miroité verticalement avec X, X, X, X.

Essentiellement, je dois pouvoir définir un bloc, de telle largeur et telle hauteur, et ensuite, définir ses répétitions et, optionnellement, ses symétries.

\begin{lstlisting}
var block = {
    lines : [ABBACDAB, CDDBDADD],
    horizontalSymmetry: true,
    verticalSymmetry: false,
    diagonalSymmetry: false
};
\end{lstlisting}